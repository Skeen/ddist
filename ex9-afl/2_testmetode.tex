For at teste koden kan man starte serveren via WebServer klassen og sende
kommandoer til den med en web browser.

Data opretholdes kun i serverens levetid.

En test kan best� af eksempelvis f�lgende Requests-URI\footnote{De listede strenge skal s�ttes efter \proc{url}'en, hvis testet i en browser, der har peget sig ind p� serveren.}:
\begin{verbatim}
   /Bank/1/getAccount?name=derp
   /Account/derp/getBalance?
   
   /Account/derp/deposit?amount=5
   /Account/derp/getBalance
   
   /Account/derp/withdraw?amount=10
   /Account/derp/getBalance
   
   /Account/derp/getName
\end{verbatim}
   
Hvis man for eksempel vil inds�tte 500 til Jimmy's konto, kalder man
\verb+getAccount(jimmy)+ ved
\begin{verbatim}/Bank/1/getAccount?name=jimmy\end{verbatim} og
bruger resultatet som \verb+<Object>+ til \verb+deposit(500)+, hvilket s�ledes
bliver:
\begin{verbatim}/Account/<Object>/deposit?amount=500\end{verbatim}

I tilf�lde af at en konto bliver efterspurgt via \verb+getAccount+, men ikke eksisterer, vil en konto p� det navn blive oprettet med et kontobel�b p� 0.0.
