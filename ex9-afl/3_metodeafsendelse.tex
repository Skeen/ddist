Serveren modtager \proc{get}-requests eller \proc{post}-requests, hvor
Request-URI'en overholder protokollen\footnote{Bem�rk: Vi har prioriteret
sanitizing af indkommende requests lavt indtil videre. V�r varlig.}. 

Serveren deler Request-URI'en op i de af protokollen specificerede dele:
\id{class}, \id{object}, \id{method} og derudover en r�kke parametre og
v�rdier (\id{arguments}).

Derefter s�ger den efter det Java klasse-objekt, der matcher
\id{class}\footnote{Dette foreg�r ved opslag i classloaderen (register over
klasser i den lokale JVM)}, 

Den fundne \texttt{klasse} erkl�rer en r�kke metoder, som ledes igennem for at
finde den \texttt{metode} der matcher \id{method}.

Serveren holder et map, hvor \id{object} er n�gle til et Java \texttt{objekt}.
Dette map initialiseres sammen med serveren og vedligeholdes ved tilf�jelse af
flere kontoer/banker (eller fjernelse).

Alts� finder serveren det \texttt{objekt}, der matcher med \id{object} i
mappet, og kalder \texttt{metoden} p� \texttt{objektet} med \id{argumenterne}. 

\id{Resultatet} returneres til klienten.

\begin{figure}[htbp]
\center
\begin{adjustbox}{max size={.8\textwidth}{.6\textheight}}
\tikzstyle{decision} = [diamond, draw, fill=blue!20,
    text width=4.8em, text badly centered, node distance=2.7cm, inner sep=0pt]
\tikzstyle{block} = [rectangle, draw, fill=blue!20,
    text width=5em, text centered, rounded corners, minimum height=4em]
\tikzstyle{blockwide} = [rectangle, draw, fill=blue!20,
    text width=10em, text centered, node distance=1.4cm, rounded corners, minimum height=2em]
\tikzstyle{line} = [draw, very thick, color=black!50, -latex']
\tikzstyle{cloud} = [draw, ellipse,fill=red!20, node distance=3.5cm,
    minimum height=2.4em]
\tikzstyle{block2} = [rectangle, draw, fill=red!20,
    text width=5em, text centered, rounded corners, minimum height=4em]
\tikzstyle{case} = [near start, color=black]

\begin{tikzpicture}[scale=2, node distance = 3.5cm, auto]

    % Place nodes
    \node [cloud] (init) 
            {Server Method Dispatch};
    \node [blockwide, below of=init] (intro)
            {Tokenize URI-Request };
    \node [decision, below of=intro] (gotClass) 
            {Does the JVM recognize the \id{class}?};
    \node [cloud, below left of=gotClass] (error) 
            {error};
    \node [block, right of=gotClass] (yClass) 
            {\texttt{Class} variable stored};
    \node [decision, below of=yClass] (gotMethod) 
            {Do the \texttt{Class} declare the \id{method}?};
    \node [block, right of=gotMethod] (yMethod) 
            {\texttt{Method} variable stored};
    \node [decision, below of=yMethod] (gotMapping) 
            {Is the \texttt{class} mapped by \id{object}?};
    \node [block, below of=gotMapping] (yMapping) 
            {\texttt{Object} retrieved from map and stored};
    \node [block, left of=yMapping] (parse) 
            {Parse \id{arguments}};
    \node [block, left of=parse] (invoke) 
            {Invoke \texttt{Method} on \texttt{Object} with \texttt{Arguments}};
    \node [block2, left of=invoke] (fin) 
            {Return \id{result} from the method call};

    % Draw edges
    \path [line] (init) -- (intro);
    \path [line] (intro) -- (gotClass);
    \path [line] (gotClass) -- node [case] {No} (error);
    \path [line] (gotClass) -- node [case] {Yes} (yClass);
    \path [line] (yClass) -- (gotMethod);
    \path [line] (gotMethod) -- node [case] {No} (error);
    \path [line] (gotMethod) -- node [case] {Yes} (yMethod);
    \path [line] (yMethod) -- (gotMapping);
    \path [line] (gotMapping) -| node [case] {No} (error);
    \path [line] (gotMapping) -- node [case] {Yes} (yMapping);
    \path [line] (yMapping) -- (parse);
    \path [line] (parse) -- (invoke);
    \path [line] (invoke) -- (fin);
\end{tikzpicture}

\end{adjustbox}
\caption{Flow diagram, der viser hvordan serveren konverterer URI-requesten til
et metodekald.}
\end{figure}
