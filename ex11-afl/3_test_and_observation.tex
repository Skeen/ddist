\section*{Test and obseration}

%TODO: Test explanation and results from MulticastQueueOrderTest
\subsection*{Testing of the ordering}
 % Develop a test which runs five instances of you implementation and simultaneously
 % and quickly puts a lot of messages at all five peers.
 % Let the test check that they arrive in the same total at all peers.
 % See MulticastQueueFifoOnlyTest.java for inspiration.
 % Your report must have a section which describes how you did the test, why this
 % is an appropriate way to do the test, and what the result of the test was.

Testen k�rt med Ant via \verb+ant junit+ gennemf�res korrekt.
Vi har testet at MMJoins og MMPayloads kommer frem i samme
r�kkef�lge\footnote{Hver test skiller imellem at teste ankomsten af MMJoins og
MMPayloads.}.


%TODO: Test explanation and results from MulticastQueuePerformanceTest
\subsection*{Testing of the performance of total ordering vs. fifo ordering}
Testen k�rt med Ant via \verb+ant testP+, gennemf�res korrekt.

Samtlige testK�rsler er vedh�ftet.
Her er et trace af vores bedste testk�rsel:
\VerbatimInput[firstline=76,frame=leftline,fontsize=\small,numbers=left,numbersep=6pt]{trace3.txt}

Vi har oplevet, ved at s�tte en maxtid p� hvor lang tid vi venter p� hvert get
(udkommenteret i koden), at nogle enkelte MMPayloads tager over 20 sekunder at
f� fat i.
Hvis vi ikke indf�rer denne max-tid, kan get forts�tte med at blokere i flere 
timer.

Det er evt. et problem med manglende portbinding.

Dette er fortsat et uafklaret problem.

% Vi har derfor udtaget en testk�rsel, hvor vi var 'heldige' at der ikke var nogle blokeringer.

Vi har ikke v�ret i stand til at finde en enkelt testk�rsel med 3 eller flere
datapunkter, n�r vi sender 100 beskeder.

Vi har pr�vet at n�jes med at sende 1 og 10 beskeder.
Resultatet er vedh�ftet i appendix.

