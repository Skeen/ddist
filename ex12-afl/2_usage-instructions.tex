\subsection*{Kompilering}
Ant can bruges til at kompilere koden. 
Derefter kan man bruge java i terminalen fra build folderen.

Dvs. man kan starte vores regne server med 'java ServerTUIDistributed' og
clienten kan startes med 'java ClientTUI'

Eksempel for at kompilere og k�re i Windows:
\begin{verbatim}
...\kode>ant build
Buildfile: ...\kode\build.xml

prepare-build:

build-src:

build-test:
    [javac] ...

build:

BUILD SUCCESSFUL
Total time: 0 seconds

...\kode>cd build

...\kode\build>java ServerTUIDistributed
Exit: Gracefully logs out
Crash: Makes the server crash.

Enter address of another server (ENTER for standalone):
\end{verbatim}

Alternativt, kan man k�re de to bash scripts 'server.sh' og 'client.sh' for at
k�re de tilsvarende TUI's.

N�r en server startes, er der en foresp�rgsel om en server-adresse.
Hvis man undlader at skrive en adresse og blot trykker ENTER, vil serveren lave
en gruppe self.

For at forbinde til en eksisterende gruppe af servere skal man indtaste
\verb+<address>:<port>+.

Da porte oprettes dynamisk, skal man v�re opm�rksom p� at forbinde til den
server port, som er blevet bundet i den kendte peer.
 
I ClientTUI bliver man ligeledes spurgt om en server adresse.
Her skal man igen bruge \verb+<address>:<port>+.
% Todo : cleanup the following (duplicated?)
Bem�rk at dette er en anden port end 'server til server' porten.


\subsection*{Server ops�tning}
Den f�rste server der startes (den der skal lave server gruppen), skal ikke
gives noget argument, ved beskeden;
\begin{verbatim}
Enter address of another server (ENTER for standalone):
\end{verbatim}
Her skal vi alts� simpelt nok bare klikke '<ENTER>'.

For en server der skal joine en server gruppe, skal vi p� givne tidspunkt skrive
server group addressen p� den server vi �nsker at joine her; Denne findes nemt
ved at l�se outputtet fra en allerede k�rende server, og ser noglelunde s�ledes
ud;
\begin{verbatim}
...
Created/Joined server group at: llama04/10.11.82.4:53261
...
\end{verbatim}
Det er netop denne addresse der skal bruges, til at blive en del af server
gruppen. Dette skal indtastes med b�de addresse og navn, p� det s�dvanlige
'<address>:<port>' format. Man skal huske at kigge efter beskeden; 'Server
running', der indikere at serveren nu faktisk k�re.

\subsection*{Klient ops�tning}
For at kontakte til en server med en client skal man s�ledes ogs� kende noget
information fra serveren, nemlig p� hvilken port, den specifikke server,
servicere clienter, denne information er ligeledes nem at l�se, og ser
noglelunde s�ledes ud;
\begin{verbatim}
...
Listening for clients at: llama04/10.11.82.4:46814
...
\end{verbatim}
I clienten skal denne addresse, skrives p� samme format, som tidligere, n�r
beskeden;
\begin{verbatim}
Enter address of a server:
\end{verbatim}
Vises p� sk�rmen, herefter vil clienten udskrive information om, hvorvidt det
lykkedes at kontakte til serveren eller ej.
Noter venligst, at client porten, er anderledes en servergroup porten.

