Vi har vedh�ftet f�lgende filer:

\begin{verbatim}
.:
kode/
report.pdf

./kode:
build.xml
src/

./kode/src:
WebServer.java
WebServerThread.java
\end{verbatim}

\subsection*{Systemkrav}
Det kr�ves at man har Apache Ant installeret.

Det anbefales at man bruger Universitetets unix-maskiner, evt. over ssh.

Universitets computere har Ant installeret, men scriptet bruger derudover ogs� 'tar' og 'wget' til at f� fat i resourcefilerne\footnote{Opgavebeskrivelsen fort�ller hvordan man finder resourcefilerne:
The files to be served can found in \href{http://www.cs.au.dk/~jbn/dDist2011/FilesToBeServed.tar.gz}.}.

Hvis man ikke har et system med 'tar' og 'wget' kan man selv hente filerne ned og pakke ud s� mappen 'FilesToBeServed' ligger ved siden af Ant's build.xml.

Ellers vil Emil gerne kompilere og levere de manglende bin�re filer.

\subsection*{Ant build-script}

Vi har brugt et Ant build-system til at g�re testene nemme.

Fra mappen 'kode' k�res 'ant help', hvorefter en r�kke targets bliver beskrevet.
Disse targets kan k�res vha. Ant s�ledes: 'ant <target>'.

Et print af Ant's targets kan ses i Appendix A, linie 3-30.

\subsection*{Test}

En normal test-session kan udf�res s�ledes:
\begin{verbatim}
[sverre@llama14:...]$ ls
build.xml  src
[sverre@llama14:...]$ ant test-firefox
...
\end{verbatim}

En anden test-session kan foreg� s�ledes:
\begin{verbatim}
[sverre@llama14:...]$ ls
build.xml  src
[sverre@llama14:...]$ ant server
[...lots of prints...]
server:
     [java] WebServer Started!
     [java] Contact this server on the IP address 10.11.82.14:40404
^Z
[sverre@llama14:...]$ bg
[1]+ ant server &
[sverre@llama14:...]$ chromium-browser 127.0.0.1:40404 &
\end{verbatim}
I den sidste session s�ttes ant til at sove (vha. \verb+C_z+) og derefter til at k�re i baggrunden (\verb+$ bg+).

Dette tillader at en vilk�rlig browser kan �bnes til at pege p� ip-addressen.
Og man kan desuden ogs� se rapporteringer fra WebServeren om modtagne requests.

\subsubsection*{Efter at serveren er startet og browseren har peget sig ind p� den.}

N�r internet-browseren er startet kan det tjekkes manuelt at hjemmesiden er oppe at k�re.

Derudover kan f�lgende ting bekr�ftes:
\begin{itemize*}
\item Html modtages og vises som hjemmesider
\item Links virker og henviser korrekt til Wikipedia
\item Side 2 virker og viser korrekt alle 3 billeder
\item Side 3 virker og referencen til pdf'en vises bliver h�ndteret korrekt af browseren.
\item Desuden bliver text-filen h�ndteret korrekt af browseren.
\end{itemize*}

F�lgende virker dog ikke, fordi det ikke er implementeret:
\begin{itemize*}
\item Side 4 bliver vist, men post-requesten, g�r ikke som beskrevet.
\end{itemize*}
