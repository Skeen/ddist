\subsubsection*{Is it possible in your implementation that different users, A
and B say, sees different orderings of messages sent by two different users, C
and D say?}

I vores system er dette en mulighed, da beskeder vises i den r�kkef�lge de
modtages i. 
Dette er mest p�virket af netv�rk-latency, da brugere kan have vidt forskellige
latency'er i forhold til hinanden.


\subsubsection*{If your system has the mentioned problem, then discuss what
could be done to fix it.}
    
En l�sning, ville her v�re at tidsstemple beskederne.
S�ledes vil man blot kunne printe beskeder i overens stemmelse med
tidsstemplerne.
Hvis man antager at samtlige clocks p� alle i systemet er perfekt synkroniseret vil dette alts� v�re muligt.
Men da dette ikke kan lade sig g�re, kan vi blot approksimere det, hvilket
alts� betyder, at en computer, der tidsstempler med $+200ms$, altid vil have
sine beskeder vist senere end dem, der er helt pr�cise.

En alternativ l�sning vil v�re at lave en mutex exclusion p� at sende beskeder
i en gruppe, s�ledes at kun en besked kan sendes i netv�rket ad gangen.

En sidste alternativ l�sning ville v�re at udr�be en 'super-peer', der ville
modtage og fordele samtlige beskeder, s�ledes at dennes r�kkef�lge af beskeder
ville blive samtlige modtageres r�kkef�lge.
