Efter at have k�rt koden, er det muligt at l�seren har lagt m�rke til nogle sm�
defekter ved programmet. 
Dem der er kendt af udviklerne er:
\begin{itemize}
\item I tilf�lde af at flere peers opretter og afbryder forbindelsen til en
MultiChat-gruppe, forekommer der gentagelser af 'join'-beskeder i historien.
\item Der er problemer, hvis man pr�ver at forbinde til en MultiChat-peer p� \verb+localhost+ eller \verb+127.0.0.1+. Brug i stedet IP-adressen som konsollen angiver ved beskeder.
\end{itemize}

Vi konkluderer at vi har l�st opgaven med vores kode i tilfredsstillende grad,
da vi kan udf�re pr�cis den chat-sekvens som er fremvist i bunden af opgavebeskrivelsen.

\subsubsection*{Late additional note}
Vi har tilf�jet forkortelsen '\verb+ant mc+' for '\verb+ant multichat+'.

Vi har haft en succesfuld testsession over vpn p� tv�rs af OS's (llamaXX og egen windows pc).
Vi opdagede dog en masse forsinkelse s� l�nge flere peers var forbundet over vpn-forbindelsen.

Derudover har vi arbejdet med et \proc{gui} og \proc{cli}, hvor der dog stadig er lidt problemer.
Man kan tjekke de forskellige Ant targets vha. '\verb+ant help+.

